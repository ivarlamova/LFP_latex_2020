\section{Abstract}
1. Cathode is the positive electrode in the batteries essential for their general electrochemical features.

Several materials have been used as a cathode part in the lithium-ion rechargeable batteries, but their electrochemical characteristics have not satisfied all user requests. 

LiFePO$_4$ and LiMnPO$_4$ materials have a high capacity, good operating voltage and cycle stability compared to other cathode polyanionic compounds.

\textit{DFT approach was used in the investigation of the hydroxyl defects structure within LiFePO$_4$ and LiMnPO$_4$ cathode material. The aim of current work is to establish the influence of OH defects on the structure of LiFePO$_4$ and LiMnPO$_4$ materials and their electronic properties. These properties need to be improved due to the poor electonic conductivity of LiFePO$_4$ and LiMnPO$_4$ materials in their pure form .}

The aim of current work is to establish the influence of OH defects on the structure of LiFePO$_4$ and LiMnPO$_4$ materials and their electronic properties. They need to be improved due to the poor electonic conductivity of LiFePO$_4$ and LiMnPO$_4$ materials in their pure form. \textit{DFT approach was used in the investigation. }

The structure of OH-defects in LiFePO$_4$ can not be determined from the experiment, therefore their computational study is required. The Density Functional Theory in the VASP code implementation is used to identify and study the defect structure of material on the atomic level. 

The current work established the equilibrium configuration of the hydroxyl group defect within LiFePO$_4$ and LiMnPO$_4$ materials. The molecular dynamic simulations were provided in order to achieve the full knowledge about the defect distribution at the certain temperatures. The thermodynamical analysis shows the appropriate of PO$_4$-O$_4$H$_4$ substitutional defect compared to the other positions. As a result, the impact of hydroxyl defect to the electronic properties of materials was established.

\textit{Investigation of the possible interstitial and substitutional hydrogen defects and potential energy surface of their possible positions in the combination with molecular dynamic simulations were provided in order to achieve the full knowledge about the defect distribution.} These defects have a huge influence on the batteries electrochemical characteristics and their study is critical for further material improvement. 

2. LiFePO$_4$ and LiMnPO$_4$ materials have a high capacity, good operating voltage and cycle stability compared to other cathodes were used in battery application.

The structure of OH-defects in LiFePO4 is highly difficult to establish from the experiment, therefore their computational study is required. The aim of current work is to establish the structure of OH defects in LiFePO$_4$ and LiMnPO$_4$ materials and their electronic properties. The structure of OH-defects in LiFePO$_4$ can not be determined from the experiment, therefore their computational study is required. The Density Functional Theory is used to identify and study the defect structure of material on the atomic level. 

The current work established the equilibrium configuration of the OH-defect in LiFePO$_4$ and LiMnPO$_4$ materials. The molecular dynamic simulations were provided in order to achieve the full knowledge about the defect distribution at the certain temperatures corresponding to synthesis conditions in the heat treatment process. The thermodynamical analysis shows the appropriate of PO$_4$-O$_4$H$_4$ substitutional defect due to its low formation energy compared to the other substitution defects like Fe-H$_2$ or Li-H exchange. As a result, the impact of hydroxyl defect to the electronic properties of materials was established. Owing to the charge compensation mechanism the electronic density appears at the Fermi level, which corresponds to the conductivity.