\section{Potential energy surface search}

Method of equilibrium hydrogen positions in phosphorus vacancy site finding is used in the current work and named Potential Energy Surface (PES) scan. It based on comparison of the formation energies of different defective structure configurations. The formation of substitution defects occurs consistently depending of region under consideration. In case of substitution phosphorus by four hydrogen atoms each of four hydrogen atoms alternately fit to the tetrahedral oxygen environment. The procedure of PES is described below.

The possible hydrogen positions distribution around the first oxygen of tetrahedron was considered. The criterion of the localization was a distance between oxygen and hydrogen is 1\AA. As a result, all possible position of hydrogen formed a spherical symmetrical cloud of hydrogen around oxygen in its superposition. Each of hydrogen position was corresponded to separated structure with phosphorus vacancy and one hydrogen. The single point calculation was provided to determine the energy of each structure. Using this values the formation energy of each hydrogen localization can be calculated using equation \ref{eq:form}.  

\begin{equation}
E_{Hform} = \frac{1}{n}[E(LiFeP_{1-x}H_nxO_4) - E(LiFePO_4) - nxE(H)]
\label{eq:form}
\end{equation}

This equation corresponds to a different number n of substitutional hydrogen, for the first iteration n is equal to 1. All possible positions of hydrogen with corresponding energy in color representation were applied in the same phosphorus deficiency crystal for illustration. Its energetic characteristic was in strongly dependence of neighboring areas: it could be the void or occupied polyhedral sites. According that knowledge, for each hydrogen position in individual spatial area the relaxation of atomic positions was provided. After that the most stable equilibrium structure of LiFePO$_4$ with substitution of phosphorus by one hydrogen was considered. This structure became the initial configuration for the further step of second hydrogen position considering around second oxygen with PES method. In this application n is equal to 2 in the equation \ref{eq:form}. This procedure was repeated two more times. Finally, all possible hydrogen configurations were considered and the equilibrium structure of LiFePO$_4$ with substitution of phosphorus by four hydrogen was found.