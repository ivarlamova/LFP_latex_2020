%\kant[1-2]


Переход к полностью электрическим транспортным средствам и возобновляемым источникам энергии в значительной степени зависит от развития литий-ионных аккумуляторов. Одна из самых перспективных технологий литий-ионных батарей базируется на использовании катодного материала LiFePO$_4$, сочетающего в себе высокую стабильность, безопасность и доступность. Однако дальнейшее совершенствование данного материала, включающее полное или частичное замещение атомов железа на марганец, требует лучшего понимания и контроля над образующимися в процессе синтеза точечными дефектами. 

Ранее экспериментально было доказано, что материал LiFePO$_4$, полученный в гидро\-термальном синтезе, может содержать неизвестные дефекты OH-типа, которые оказывают пагубное влияние на электрохимические свойства. Однако непосредственное наблюдение структуры дефекта OH в эксперименте является довольно сложной задачей. Поэтому, в целях преодоления экспериментальных трудностей, в рамках настоящей работы используются основанные на теории функционала плотности методы, направленные на изучение структуры и энергетики OH дефектов в LiFePO$_4$ и LiMnPO$_4$ катодных материалах.

Были рассмотрены все возможные OH дефекты, включая примесные атомы H в пустотах структуры, либо дефекты замещения атомов Li, Fe, или P на атомы H. Изучение энергии формирования таких дефектов определило преимущественность дефекта PO$_4$/(OH)$_4$ по сравне\-нию с другими расмотренными, что объяснило экспериментальные результаты. Проведенное полное сканирование поверхности потенциальной энергии позволило установить, что атомная структура дефекта PO$_4$/(OH)$_4$ аналогична дефекту OH в структуре минералов на основе крем\-ния. Кроме того, динамика данного дефекта в структуре изучалась при повышенных температурах с помощью \textit{ab initio} метода молекулярной динамики.

