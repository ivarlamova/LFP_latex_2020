\chapter*{Conclusions}
\addcontentsline{toc}{chapter}{Conclusions}
The study of OH defects in LiFePO$_4$ and LiMnPO$_4$ compounds was performed with density functional theory (DFT) and ab initio molecular dynamics. The energy and structure of interstitial H as well as substitutional Li/H, Fe/2H, and PO$_4$/(OH)$_4$ defects were calculated with DFT and DFT+U methods. It is shown that the DFT+U method is required for the correct reproduction of electronic structure and formation energies, while DFT without U method can be used for calculating the structure of defects.

In the case of PO$_4$/(OH)$_4$ defect full scan of potential energy surface was performed and allowed to find the lowest-energy structural configurations of this defect.
According to the developed thermodynamic model and DFT+U results the PO$_4$/(OH)$_4$ defect has the lowest formation energy per hydrogen atom among all considered defects.
In PO$_4$/(OH)$_4$ defect four hydrogen atoms form covalent O--H bonds with 1~{\AA} bond length. Three O--H bonds are directed inside the O$_4$-tetrahedral void left after phosphorus atom removal, while the fourth O--H is located in the neighborhood tetrahedral void.

The \textit{ab initio} molecular dynamics showed that the PO$_4$/(OH)$_4$ defect remains localized at temperatures up to 1100\textdegree C emphasizing its strong trapping to P vacancy. At the same time, local rearrangements of H atoms inside the defect are observed at much lower temperatures. 
In particular, the temporal formation of H$_2$ molecule with a bimodal distribution of H-H bond length (0.7 and 0.85~{\AA}) was observed inside PO$_4$/(OH)$_4$ defect at 600\textdegree C. At 1100\textdegree C the temporal formation of H$_2$O molecule with distorted H-O-H angles inside PO$_4$/(OH)$_4$ defect was also discovered.

Overall, the current study provides insights into the structure and dynamics of OH defects, which is required for further studies of their influence on materials properties. The revealed dependencies of OH formation energies on oxygen chemical potential may help to rationalize the synthesis procedure of LiFePO$_4$ cathode material and reduce the concentration of detrimental OH defects. 

% At 600\textdegree C temporal formation of H$_2$ complexes inside  LiFePO$_4$ with 6.25\%  PO$_4$/(OH)$_4$ defects concentration was observed. The H--H bond length of H$_2$ molecule with 0.7\AA \xspace corresponds to its gas phase state.

% The density of states calculation shows the presence of the permitted electronic states at the Fermi level. The conductivity presence is confirmed, which can make an influence on the batteries performance. 

