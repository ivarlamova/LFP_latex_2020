%\kant[1-2]

The transition to all-electric vehicles (EV) and renewable energy heavily relies on the further development of Li-ion batteries. One of the most prospective Li-ion technology is based on LiFePO$_4$ cathode material,  which combines high stability, safety, and affordability. However, the further improvement of this material, including full or partial replacement of Fe by Mn for an increase of voltage, requires better understanding and control of residual point defects left after synthesis. 

Recently, it is shown experimentally, that hydrothermally synthesized LiFePO$_4$ cathode material can contain previously unknown OH-type defects, which have a detrimental effect on electrochemical properties. However, the direct observation of OH defect atomic structure was proved to be highly challenging with experimental techniques. Therefore, in the current work to overcome experimental difficulties we employ computational methods based on density functional theory to study the energetics and structure of OH defects in LiFePO$_4$ and LiMnPO$_4$ cathode compounds.

We consider all possible OH defects, where H resides either in interstitial voids or substitute Li, Fe, or P. The calculation of defects formation energies shows that PO$_4$/(OH)$_4$ substitution is the most favorable among considered defects, explaining experimental results. Be performing a full potential energy surface scan it is found that the atomic structure of PO$_4$/(OH)$_4$ defect is similar to the hydrogarnet defect observed in silicon-based minerals. Finally, the dynamics of this defect is studied at elevated temperatures with \textit{ab initio} molecular dynamics. 

% and it is shown that hydrogen atoms remains attached to the P vacancy up to 1400~K, while their local rearangment 




% The aim of the current work is to investigate the structure and energetics of recently discovered OH defects in LiFePO$_4$ and LiMnPO$_4$ compounds.

%  The structure of OH-defects in LiFePO$_4$ is highly difficult to detect from the experiment, therefore their computational study is required. The Density Functional Theory is used to identify and study the defect structure of material on the atomic level. 

% and the dynamics of these defects at the elevated temperatures.
% The current work considered the several types of possible defects: the interstitial hydrogen in the voids of structure as well as substitution of Li/1H, Fe/2H or P/nH (n$\in$[1, 5]). The formation energies of such defects were calculated. This analysis shows the appropriate of PO$_4$/(OH)$_4$ substitutional defect due to its low formation energy compared to the other defects. Consequently, the equilibrium configuration of the OH-defect in the phosphorus deficiency LiFePO$_4$ and LiMnPO$_4$ was established. 

% The molecular dynamic simulations were provided in order to achieve the full knowledge about the defect evolution over time at elevated temperatures. The structure under consideration had either 25\% or 6.25\% of PO$_4$/(OH)$_4$ defect concentration. The temperature of molecular dynamics was equal to 600\textdegree C corresponding to synthesis conditions in the heat treatment process or 1100\textdegree C of the annealing process. After this simulations two more stable low energy configurations of hydroxyl PO$_4$/(OH)$_4$ defects in LiFePO$_4$ and LiMnPO$_4$ were identified.

